
The practical gains of regression testing are widely discussed (e.g., \citep{aiken1991multiple,leung1989insights,wong1997study,ekelund2015efficient}). In the context of agile development, this testing strategy plays an important role by working as safety nets changes are performed \citep{martin2002agile}. \citet{parsons2014influences} investigate regression testing strategies in agile development teams and identify factors that can influence the adoption and implementation of this practice. They found that the investment in automated regression testing is positive, and tools and processes are likely to be beneficial for organizations. Our strategies (distance functions, distance functions and machine learning) are automatic ways to enable the preservation of regression test cases. 

\citet{ali2019enhanced} propose a test case prioritization and selection approach for improving regression testing in agile projects. Their approach prioritizes test cases by clustering the ones that frequently change. Here, we see a clear example of the importance of preserving test cases. 
%Our approaches help to avoid test case discards. }

Some work relates agile development to Model-Based Testing, demonstrating the general interest in these topics. \citet{katara2006making} introduce an approach to generate tests from use cases. Tests are translated into sequences of events called action-words. This strategy requires an expert to design the test models. \citet{puolitaival2008adapting} present a study on the applicability of MBT in agile development. They refer to the need for technical practitioners when performing MBT activities and specific adaptations. \citet{katara2006making} discuss how MBT can support agile development. For that, they emphasize the need for automation aiming that MBT artifacts can be manageable and with little effort to apply.

\citet{cartaxo2008lts}
 propose a strategy/tool for generating test cases from ALTS models and selecting different paths. Since the ALTS models reflect use cases written in natural language, the generated suites encompass the problems evidenced in our study (a great number of obsolete test cases), as the model evolves.

\citet{de2016full} discuss a series of problems related to keeping MBT suites updated during software evolution. To cope with this problem, they propose a test selection approach that uses test case similarity as input when collecting test cases that focus on recently applied changes. Oliveira Neto et al.'s approach refers to obsolete all test cases that are impacted in any way by edits in the requirement model. However, as our study found, a great part of those tests can be little impacted and could be easily reused, avoiding the discard of testing artifacts. 

The test case discard problem is not restricted to CLARET artifacts. Other similar cases are discussed in the literature (e.g.,\citep{de2016full,nogueira2007model}). Moreover, this problem is even greater with MBT test cases derived from artifacts that use non-controlled language \citep{pinto2012understanding}.

Other works also deal with test case evolution (e.g., \citep{katara2006making,pinto2012understanding}). They discuss the problem and/or propose strategies for updating the testing code. Those strategies do not apply to our context, as we work with MBT test suite evolution generated from use case models. 

Distance functions have been used in different software engineering scenarios (e.g., \citep{runkler2000automatic,okuda1976method,lubis2018combination}). For instance, \citet{runkler2000automatic} use the Levenshtein function to automatically extract keywords from documents. In the context of MBT, \citet{coutinho2016analysis} investigated the effectiveness of a series of distance functions when used combined with strategies for suite reduction based on similarity. Although in a different context, their results go according to ours where all distance functions performed in a similar way.

The use of machine learning techniques in software engineering is not new. \citet{baskeles2007software} propose a model for estimating development effort aiming at overcoming problems related to budget and schedule extension. \citet{gondra2008applying} uses an artificial neural network to determine the importance of software metrics for predicting fault-proneness. 

\citet{durelli2019machine} present a systematic mapping study on machine learning applied to software testing. From 48 selected primary studies, they found that machine learning has been used mainly for test-case generation, refinement, and evaluation. For instance, \citet{strug2012machine} use a KNN-learner to reduce the set of mutants to be executed in mutation testing. It predicts when a test can kill certain mutants. \citet{fraser2015assessing} propose an approach based on machine learning algorithms to evaluate test suites using behavioral coverage. It receives data from a test generation tool and predicts the behavior of the program for the given inputs. \citet{zhu2008experience} propose a model for estimating test execution efforts based on testing data such as the number of test cases, test complexity, and knowledge of the system under testing. \citet{chen2011using} present a machine learning approach for selection regression test cases. Their learner clusters similar test cases based on an input function and constraints. Our work differs from the others since it uses machine learning and distance functions to predict the impact of a given use case update and to avoid the discard of MBT test cases. 

%%%%%%%%
 

%Silva et al. \cite{Silva:2018:SIM:3266003.3266009} gave the first steps on investigating issues related to MBT in the context of agile development. They ran an empirical study on the evolution of specification models and their impact on generated MBT suites and found that 86\% of a test suite is often impacted, however, more than half those tests were impacted due to syntactic model edits (low impact). Those findings greatly motivated this current research. Based on them, here we propose the use of distance functions to automatically classify the test cases that are little impacted and could be reused.








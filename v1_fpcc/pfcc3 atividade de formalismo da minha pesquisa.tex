\documentclass{article}
\usepackage{amsmath, amssymb}

\begin{document}

\title{Formalismo Matemático para Utilização de TTS na Leitura Humanizada de Textos e seus Impactos na Acessibilidade e Entendimento}
\author{Thomaz Diniz Pinto de Morais}
\date{Junho 2024}
\maketitle

\section{Introdução}

A pesquisa investiga a utilização de TTS (Text-to-Speech) para a leitura humanizada de textos e os impactos dessa tecnologia na acessibilidade e no entendimento dos usuários. Neste documento expressaremos 3 formalismos matemáticos distintos. Dois destes formalismos foram feitos em atividades anteriores, mas que também são relevantes para minha pesquisa e um deles foi adicionado para o caso desta repetição não ser o suficiente para se adequar a atividade da disciplina. Estes são os formalismos que são relevantes para a pesquisa que irei conduzir:


\begin{itemize}
    \item Formalização de como o TTS que utiliza um autocoder variacional funciona (Primeira atividade de FPCC3)
    \item Formalização de parâmetros objetivos para definir se um TTS é melhor que outro (Segunda atividade de FPCC3)
    \item Formalização de parâmetros subjetivos para definir se um TTS atingiu um nível de fala humanizada (Uma terceira formalização para o caso das duas anteriores não serem o suficiente para a atividade)
\end{itemize}

\section{Formalização de um TTS que utiliza um autocoder variacional}

A formalização a seguir explica um autoencoder variacional. O objetivo de um autoencoder variacional é codificar um input automaticamente em uma representação comprimida que pode depois ser decodificado para que o input possa ser reconstruído a partir do que foi codificado. Isso pode ser usado para explicar o processo de reconstrução de voz a partir de um treinamento utilizando este autoencoder. Em um primeiro momento é feita a codificação em algo que é chamado de vetor latente Z. Em um momento do artigo é feita a formalização matemática do autoencoder para explicar seu funcionamento.

			$X$: Voz original
			
			$Y$: Sequência de texto
			
			$q(z \mid x)$: Encoder que parametriza a distribuição dos dados X para a a variável latente Z.
			
			$p(x \mid z)$: Decoder que reconstrói o input que foi passado que no caso é o audio passado.
				
			$p(z \mid y)$: Um segundo encoder que codifica uma sequência de texto Y em fonemas. E será passado ao Decoder posteriormente
			
			$S: p(z \mid y) \rightarrow  p(x \mid z)$: Voz Sintetizada a partir do processo de decodificação onde em um primeiro momento se obtém o latente Z a partir do texto, para depois ser feita a decodificação do spectograma que é utilizado em um segundo momento para formar a voz em si.
		


\section{Formalização de parâmetros objetivos para definir se um TTS é melhor que outro}
Dado que temos um modelo de TTS, podemos avaliar o quão bom é um modelo. Um modelo é considerado melhor se minimiza o número de palavras ignoradas e erros de pronúncia. Seja \( N \) o número total de palavras no texto original e \( I \) o número de palavras ignoradas pelo modelo TTS. A taxa de palavras ignoradas \( E_I \) é dada por:

\begin{equation}
E_I = \frac{I}{N}
\end{equation}

Seja \( E \) o número de pronúncias errôneas detectadas na saída do modelo TTS. A taxa de pronúncias errôneas \( E_P \) é dada por:
\begin{equation}
E_P = \frac{E}{N}
\end{equation}

O que nos dá um percentual de palavras pronunciadas erroneamente (\( E_P \)) e um percentual de palavras ignoradas (\( E_I \)). O ideal em um sistema de conversão texto-fala é que, para cada palavra no texto de entrada, não exista uma palavra com erro de pronúncia nem uma palavra ignorada. Formalmente, podemos expressar isso da seguinte maneira:

Seja \( \mathbf{X} = \{x_1, x_2, \ldots, x_N\} \) o conjunto de palavras no texto original. O sistema TTS deve garantir que nenhuma palavra seja ignorada, ou seja,
\begin{equation}
I = 0
\end{equation}
o que implica que:
\begin{equation}
E_I = \frac{I}{N} = \frac{0}{N} = 0
\end{equation}

Bem como para toda palavra, o sistema deve garantir que nenhuma palavra seja pronunciada erroneamente, ou seja,
\begin{equation}
E = 0
\end{equation}
o que implica que:
\begin{equation}
E_P = \frac{E}{N} = \frac{0}{N} = 0
\end{equation}

Portanto, o sistema TTS ideal é aquele para o qual:
\begin{equation}
E_I = 0 \quad \text{e} \quad E_P = 0
\end{equation}

Ou, em outras palavras, é desejável que um sistema TTS tenha uma quantidade de erros de pronúncia e erros de palavras ignoradas iguais a zero. E uma maneira objetiva de avaliar se um TTS é melhor que o outro é quanto o TTS sendo avaliado minimiza estes dois tipos de erros.

\section{Formalização de parâmetros subjetivos para definir se um TTS atingiu um nível de fala humanizada}

A leitura humanizada pode ser classificada como uma leitura que possui os áudios com entonação, ritmo e pausas naturais que imitam a fala humana. tan et al \cite{tan2022naturalspeech} define que um bom parâmetro para se ter uma leitura humana é utilizar uma avaliação por falantes nativos da lingua com avaliações do tipo MOS(mean opinion score) com a escala likert, comparando com leitores nativos. Se as médias entre os leitores nativos e o sistema TTS forem estatisticamente iguais, então o TTS é capaz de fazer uma leitura humanizada (ou bem próximo da qualidade humana). Portanto, há margem de fazer esta formalização também.

Utilizamos uma escala Likert de 1 a 5 para as avaliações, onde 1 representa a pior qualidade e 5 a melhor qualidade.

Sejam $M_H$ a média dos escores MOS para falantes humanos nativos e $M_T$ a média dos escores MOS para o sistema TTS. Nosso objetivo é comparar $M_H$ e $M_T$.

Formulamos as hipóteses estatísticas:

\begin{itemize}
    \item $H_0$: $M_H = M_T$ (As médias são estatisticamente iguais)
    \item $H_a$: $M_H \neq M_T$ (As médias são estatisticamente diferentes)
\end{itemize}


A partir destas hipóteses conduziremos nosso teste T de student para testar se as médias são ou não iguais utilizando o $\alpha = 0.05$.


\begin{equation}
t = \frac{M_H - M_T}{\sqrt{\frac{s_H^2}{n_H} + \frac{s_T^2}{n_T}}}
\end{equation}

onde $s_H$ e $s_T$ são os desvios padrões das amostras de falantes humanos e TTS, e $n_H$ e $n_T$ são os tamanhos das amostras de humanos e de TTS, respectivamente. Com isso encontraremos a estatística $t$ e com esse valor Calculamos o valor-p associado ao teste t onde $p-valor = P(|t| < T)$ que nos dará um valor maior ou menor que $\alpha$. Se o p-valor for menor que $\alpha$, rejeitamos a hipótese nula $H_0$; caso contrário, não rejeitamos $H_0$.

\bibliographystyle{plain}
\bibliography{ref}

\end{document}
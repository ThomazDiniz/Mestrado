\documentclass[a4paper,12pt]{article}
\usepackage[utf8]{inputenc}
\usepackage{url}
\usepackage{hyperref}
\usepackage[english,brazil]{babel}

\usepackage{bookmark}

\title{Estratégias de Sintetização de Voz utilizando Deep Learning para Leitura Humanizada de Textos}
\author{
  Aluno: Thomaz Diniz\\
  \texttt{thomaz.morais@ccc.ufcg.edu.br}
  \and
  Orientador: Herman Martins\\
  \texttt{hmg@computacao.ufcg.edu.br}
}
\date{}


\begin{document}	
	\maketitle
	\selectlanguage{english}
	\begin{abstract}
		Text-to-speech (TTS) is an extremely important tool in the pursuit of accessibility. It benefits not only people with visual impairments but also those with difficulties in reading comprehension. Recent advancements in deep learning enable the synthesis of voice with quality very close to recordings made by humans. In this study, we aim to replicate state-of-the-art techniques used to synthesize voices in English within the context of Brazilian Portuguese.
		
		Keywords: Deep learning, Acessibility
	\end{abstract}


	\selectlanguage{brazil}
	\begin{abstract}
		A síntese de fala a partir de texto é uma ferramenta crucial para promover a acessibilidade. Não apenas beneficia pessoas com deficiência visual \cite{}, mas também aquelas com dificuldades na compreensão de textos \cite{wood2017readingaccessibility}. Avanços recentes no campo do deep learning têm possibilitado a geração de voz sintética com qualidade comparável à de gravações humanas \cite{tan2022naturalspeech}. Este estudo visa replicar técnicas de estado da arte em deep learning utilizadas na síntese de voz em inglês para o contexto do português brasileiro.
		
		Palavras-chave: Deep learning, Acessibilidade
	\end{abstract}


	\selectlanguage{english}
	
	\section{Introdução}
	
		Avanços recentes na área de deep learning têm possibilitado o desenvolvimento de sistemas de síntese de voz humana de maneira cada vez mais precisa e natural. Tan et. al. por exemplo, produziram o NaturalSpeech, um modelo capaz de replicar a voz humana com uma qualidade idêntica a de gravações humanas \cite{tan2022naturalspeech}). Essa tecnologia oferece uma possibilidade de melhoria na acessibilidade digital. Recursos de acessibilidade de sintetização de voz a partir de textos (do inglês text-to-speech ou simplesmente TTS) são utilizadas a bastante tempo \cite{}, mas ainda existe uma lacuna no impacto que ferramentas que se aproximam mais a dicção e a voz humana podem trazer na melhoria da qualidade de vida para pessoas com deficiência visual e dificuldade de compreensão de textos. Neste estudo pretendemos replicar técnicas de TTS já utilizadas no contexto da lingua inglesa, mas no português e entender os impactos em utilizar técnicas de ponta para a qualidade de vida de pessoas que necessitam desta ferramenta.
	
		\subsection{Justificativa}
		A acessibilidade é crucial para inclusão e igualdade de oportunidades. No entanto, muitas pessoas enfrentam barreiras significativas ao tentar acessar informações devido a deficiências visuais, dislexia ou outras dificuldades de leitura. Embora tecnologias de TTS existam há décadas, a qualidade e naturalidade da síntese de voz tradicionalmente deixam a desejar, dificultando a compreensão e o engajamento do usuário. Por isso, pretendemos fazer um estudo em volta de uma síntese de voz humanizada. Ou seja, uma experiência de leitura mais próxima da experiência de leitura humana, com entonação, ritmo e ênfase semelhantes à fala humana.


		\subsection{Objetivo}
		Este artigo propõe explorar o potencial da síntese de voz humanizada para melhorar a acessibilidade digital, utilizando técnicas avançadas de deep learning. Buscamos desenvolver sistemas TTS para reproduzir com precisão a entonação, o ritmo e outras características da fala humana.

		\subsection{Questões de Pesquisa}
		\begin{itemize}
			\item Quais são as técnicas mais eficazes de deep learning para desenvolver TTS em português mais humanizada?
			\item Como validar o quão humanizada está a leitura do texto obtida pelo TTS?
		\end{itemize}
	\section{Revisão Sistemática da Literatura}
		\subsection{Artigos Selecionados}
			\subsubsection{Artigo 1: TACOTRON: TOWARDS END-TO-END SPEECH SYNTHESIS}
			
			Contexto e contribuições:
						
			O contexto do artigo é de text-to-speech (tts) ou síntese de voz a partir de texto. Nele é difinida uma arquitetura onde o modelo recebe caracteres como input e retorna um espectograma. A partir deste espectograma é utilizada uma outra técnica chamda de WaveNet \cite{oord2016wavenet} para transformar o espectograma em audio utilizando uma reconstrução Griffin-Lim.
			
			Como a solução proposta pelo artigo foi avaliada?
			
			A solução foi avaliada através de uma survey de falantes nativos com Mean Opinion Score (MOS) comparando com outros modelos de TTS: O modelo Parametric \cite{zen2016parametric} e o modelo Concatenativa \cite{goncalvo2016concatenative}.
			
			Pontos fortes do artigo:
			\begin{itemize}
				\item Os modelos são bem claros, e o processo é bem explicado.
			\end{itemize}
			Pontos que o artigo poderia melhorar:
			\begin{itemize}
				\item Artigo não acompanha código.
				\item Artigo não acompanha modelo para facilitar a validação.
			\end{itemize}
			
			
			\subsubsection{Artigo 2: Conversão Texto-Fala para o Português Brasileiro Utilizando Tacotron 2 com Vocoder Griffin-Lim}
			
			Contexto e contribuições:
			
			Neste paper há uma aplicação do tacotron, mas dessa vez com foco no português. O artigo consegue aplicar o português, contudo o resultado é bem aquem do esperado com diversas vozes extremamente robóticas. \cite{rosa2021ttsptbr}

			Como a solução proposta pelo artigo foi avaliada?
			
			Os autores decidiram por fazer uma avaliação de quantidade de erros de pronúncia e palavras puladas. No artigo não há uma definição do que são palavras que foram puladas mas assumo que sejam palavras cujo modelo não foi capaz de pronunciar e simplesmente ignorou.
			
			Pontos fortes do artigo:
			\begin{itemize}
				\item O paper acompanha de um github com códigos, ou seja pode ser validado e replicado com certa facilidade.
			\end{itemize}
			Pontos que o artigo poderia melhorar:
			\begin{itemize}
				\item A validação parece ser uma estratégia boa para quando não se tem muitos recursos, mas ela é extremamente limitada a um modelo que espera-se que cometa erros muito fortes.
				\item Resultados das vozes geradas não são muito bons.
			\end{itemize}
			
			\subsubsection{Artigo 3: NaturalSpeech: End-to-End Text-to-Speech Synthesis with Human-Level Quality}
			
			Contexto e contribuições:
						
			O contexto também é de TTS, além de trazer a proposta de uma técnica de sintetização de voz, ele também se propõe a trazer uma maneira de definir e julgar o quão próximo o sintetizador de voz está do nível de qualidade de voz humana. E no artigo é dito que eles conseguem algo muito próximo da qualidade humana definindo um sintetizador de qualidade humana como: Um sintetizador que gera vozes em que não há diferença estatística entre uma gravação humana e a síntese de voz pela primeira vez em uma mean opinion score.
			
			Como a solução proposta pelo artigo foi avaliada?
			
			A solução foi avaliada através de uma survey de falantes nativos com escala likert comparando com outros modelos de TTS: O modelo Parametric \cite{zen2016parametric} e o modelo Concatenativa \cite{goncalvo2016concatenative}.
			
			Pontos fortes do artigo:
			\begin{itemize}
				\item Acompanha código o que facilita na replicação.
			\end{itemize}
			Pontos que o artigo poderia melhorar:
			\begin{itemize}
				\item Muitos termos específicos da área, o que faz com que o artigo seja um pouco inacessível para alguém que está começando os estudos.
			\end{itemize}
			
	
	\section{Metodologia}
		\subsection{Tipo de Pesquisa}
		O tipo de pesquisa adotado que seguiremos serão os de experimentação e comparação. Onde em um primeiro momento testaremos múltiplas técnicas de síntese de voz, tentando entender os desafios para aplicá-las no contexto do português brasileiro e em um segundo momento realizaremos a comparação dos resultados.
		
		\subsection{Estudos Selecionados}
		Alguns dos estudos selecionados estão contidos na seção de revisão sistemática da literatura. Atualmente o Estado da arte é o NaturalSpeech, realizaremos experimentações com os modelos de deep learning do tacotron e de outros estudos também.
		
		\subsection{Extração dos dados}
		Para popular os nossa base de dados de vozes, utilizaremos o Fala Brasil \cite{falabrasil}, mas estamos em busca de outros bancos de vozes disponíveis gratuitamente online para este momento preliminar da pesquisa.

		\subsection{Validação}
		Parte dos nossos estudos iniciais vão ser para entender como validar as nossas experimentações. Alguns dos estudos que já mapeamos utilizam MOS(mean opinion score ou Pontuação média de opinião no português), que são parâmetros subjetivos feitos através de pesquisas com falantes nativos da linguagem. Gostaríamos de estudar também possibilidades de algoritmos de comparação entre a voz sintetizada e um baseline de voz humana gravada.
	
	\bibliographystyle{ACM-Reference-Format}
	\bibliography{ref}

\end{document}